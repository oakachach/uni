\documentclass[spanish]{article}

%%%%%%%%%%%%%%%%%%%%%%%%%%%%%%%%%%%%%%%%%%%%%%%%%%%%%%%%%%%%%%%%%%%%%%%%%%%%%%%%

% Language
\usepackage[spanish]{babel}

% Support for images
\usepackage{graphicx}

% Underlining
\usepackage{amsmath}

% Avoiding indenting of first paragraph's line.
\setlength{\parindent}{0cm}

% Support for hyperlinks.
\usepackage{hyperref}
\hypersetup {
        linktoc=all,
        hidelinks
}

% Additional section formatting.
\renewcommand\thesection{\arabic{section}}
\renewcommand\thesubsection{\thesection.\arabic{subsection}}

% Cover of the document
\title{Ingeniería de la usabilidad - Reto 3}
\author{Oussama Akachach Jouhrati\\[0.5cm]{\small Profesor/a: Mireia Ribera Turró}}
\date{\today}

%%%%%%%%%%%%%%%%%%%%%%%%%%%%%%%%%%%%%%%%%%%%%%%%%%%%%%%%%%%%%%%%%%%%%%%%%%%%%%%%

\begin{document}

\pagenumbering{gobble}
\maketitle
\newpage

\tableofcontents
\pagenumbering{arabic}
\setcounter{page}{2}
\newpage


% Customize from here.

\begin{abstract}

En este reto hemos estado trabajando en la planificación del
test con usuarios que se realizará en el siguiente reto.
Aquí, hemos realizado todos los pasos necesarios para
planificar el test de usuarios para evaluar la usabilidad de
la web de Freixenet.\newline

Inicialmente, se han entregado los criterios de inclusión y los
criterios de exclusión para escoger a los usuarios de
nuestro test. Seguido de ello, se han listado las métricas
que usaremos para evaluar el test, relacionadas con la
eficiencia, eficacia y satisfacción y se ha razonado la
inclusión de dichas métricas.\newline

Por último, hemos escogido las tareas que realizarán
nuestros usuarios en el test, con una descripción para el
usuario y una descripción para los entrevistadores o
entrevistadoras. También se han escrito los documentos que
ayudarán a la correcta realización del test, como el guión
del test, el documento de consentimiento informado, las
hojas de las descripciones de lsa tareas, el documento de la
encuesta SUS y las hojas de recogida de datos para cada
tarea.\newline

Como actividad complementaria, se ha realizado una lectura
opcional y se ha escrito una entrada en el foro
correspondiente a la tarea.

\end{abstract}
\newpage

\section{Pregunta 1}

\textit{Entrega los criterios de inclusión y los criterios
de exclusión.}

\subsection{Criterios de inclusión}

\begin{itemize}

\item Ser mayor de edad (más de dieciocho años).

\item Saber utilizar un navegador web de manera cómoda.

\item Beber alcohol. En concreto, cava o vino.

\item Tener interés en el mundo de la enología.

\item Conocer la marca Freixenet.

\end{itemize}

\subsection{Criterios de exclusión}

\begin{itemize}

\item Tener discapacidades visuales severas (ceguera total o
parcial, visión borrosa, dificultad de distinción de
colores...).

\item tener una discapacidad motora que impida al/la
usuario/a mover las manos.

\item Tener trastornos de déficit de atención.

\item Haber visitado la página recientemente.

\end{itemize}

\section{Pregunta 2}

\textit{Entrega el listado de las métricas, relaciona cada
una de ellas con eficiencia, eficacia y satisfacción y
razona su inclusión brevemente (1-2 líneas por
métrica).}\newline

Consideramos como error:

\begin{itemize}

\item Elegir una página equivocada para la función
requerida.

\item Introducir datos erróneos en los formularios.

\item Tardar más del tiempo requerido para terminar la
tarea.

\end{itemize}

Este test de usabilidad se focalizará en:

\begin{itemize}

\item \textbf{Completar una transacción.} Por lo tanto, buscaremos
que la tarea resulte lo más sencilla posible.

\item \textbf{Navegación.} Por lo que buscamos que la organización de
la información sea lo más eficiente posible.

\item \textbf{Descubrir problemas.} Puesto que queremos descubrir los
diferenes casos de error que hay en nuestro producto, para
resolverlos posteriormente.

\end{itemize}

\subsection{Porcentaje de éxito por tarea}

\textbf{Relación:} Eficacia.\newline

\textbf{Razón:} Necesitamos saber cuántas veces estamos consiguiendo
éxitos en nuestras tareas para hacernos una idea de la
eficacia general de nuestro producto.

\subsection{Tiempo medio de finalización de una tarea}

\textbf{Relación:} Eficiencia y satisfacción.\newline

\textbf{Razón:} Queremos que el/la usuario/a encuentre su objetivo,
pero también buscamos que lo haga en un tiempo razonable. De
lo contrario, este/a se puede frustrar y abandonar la
página.

\subsection{Ratio de errores por tarea}

\textbf{Relación:} Eficacia, satisfacción.\newline

\textbf{Razón:} Los/las usuarios/as no esperan encontrar
errores durante la ejecución de las tareas. Buscamos hacer
la experiencia lo más intuitiva y fluida posible.

\subsection{Expresión de descontento por tarea}

\textbf{Relación:} Satisfacción.\newline

\textbf{Razón:} Aunque la página ya se ha centrado en
proveer una experiencia satisfactoria visualmente, buscamos
que los/las usuarios/as compartan dicha emoción.

\newpage

\section{Pregunta 3}

\textit{Entrega las tres tareas indicando para cada una la
descripción para el usuario y la descripción interna.}

\subsection{Tarea 1: Compra el cava Reserva Real}

\textbf{Contexto:} Es el cumpleaños de tu padre. Sabes que
se está interesando mucho por la enología y te gustaría
regalarle un detalle que refuerce su interés.\newline

\textbf{Descripción para el usuario:} Entra en la página
``freixenet.es'' y busca la página de compra del Cava
Reserva Real.\newline

\textbf{Descripción interna:} El/la usuario/a debe entrar en
la página desde el buscador del navegador web
duckduckgo.com. Seguido de ello, deberá buscar en el menú la
opción ``Cavas emblemáticos''. Consiguientemente, el/la
usuario/a deberá buscar en el listado de cavas la opción
``Reserva Real''. Por último, el/la usuario/a tendrá que
pulsar en el botón ``Comprar'' para finalizar la
tarea.\newline

Cualquier acción que no ocurra en este orden será
considerada como fallo a anotar y, en caso de que haya
expresiones verbales negativas por parte del/la usuario/a,
se anotará dentro de la métrica correspondiente.\newline

La tarea debe realizarse en menos de un minuto y treinta
segundos.

\subsection{Tarea 2: Haz una reserva para una cata de cavas
y quesos}

\textbf{Contexto:} Tu aniversario de pareja se acerca y te apetece hacer
un plan más sofisticado con tu pareja. Estás pensando en
aprovechar la correspondencia de la Renfe con la casa
Freixenet y llevarla a hacer una cata este fin de
semana.\newline

\textbf{Descripción para el usuario:} Entra en la página
``freixenet.es'' y haz una reserva para una cata de cavas y
quesos. La reserva será a nombre de Pablo Vallecas y pondrás
el correo pablovalle8756@gmail.com. Esta reserva será para
dos adultos, de horario de mañana. La primera preferencia
de idioma será el catalán y, la segunda, el castellano. La
fecha de visita es para el 5 de diciembre de 2023. El resto
de información se puede añadir de la manera que tú creas más
conveniente.\newline

\textbf{Descripción interna:} El/la usuario/a debe entrar a
la página desde el buscador google.com. Seguido de ello,
deberá buscar en el menú la opción ``Enoturismo -
Actividades enoturísticas''. Consiguientemente, el/la
usuario/a deberá acercar el ratón al cuadro con el título
``Cata de cavas y quesos'', y aparecerá visible el botón
``Información y reservas'', que deberá pulsar para que se
abra el formulario de reservas. Aquí, deberá introducir
todos los datos que se le ha indicado y, para los que no,
deberá responder de una manera coherente al tipo de campo
seleccionado. La tarea finalizará cuando pulse el botón
``enviar'' y aparezca el mensaje de confirmación en la
pantalla del formulario.\newline

La duración de esta tarea es de 5 minutos.\newline

Cualquier acción que no ocurra en este orden será
considerada como fallo a anotar y, en caso de que haya
expresiones verbales negativas por parte del/la usuario/a,
se anotará dentro de la métrica correspondiente.\newline

Además, por cada elemento del formulario escrito de manera
incorrecta según el tipo de campo contará como dos errores.
De la misma manera, un error que detecta el formulario
contará sólo como un error.\newline

En esta tarea se debe prestar especial atención a cómo el/la
usuario/a introduce el campo ``Fecha de la visita''. Si
pregunta sobre el formato o no entiende cómo introducir ese
campo, lo contaremos como expresión negativa del/la
usuario/a.

\subsection{Tarea 3: Infórmate sobre la historia de la marca}

\textbf{Contexto: } Tantos anuncios de la Renfe sobre la
casa Freixenet te han despertado la curiosidad sobre cómo
son exactamente y cómo se formó la empresa.\newline

\textbf{Descripción para el usuario:} busca, dentro de la
página, información sobre cómo se formó la empresa y cuál es
su historia.\newline

\textbf{Descripción interna:} El/la usuario/a deberá entrar
a la página desde el navegador bing.com. Seguido de ello,
deberá navegar por el menú, hasta encontrar el enlace
``Descubre Freixenet - Una historia de excelencia''. Una vez
dentro, si el usuario considera que ya ha llegado, o pulsa
en un enlace que lleve a más detalles sobre los
acontecimientos de ese año, consideraríamos la tarea como
terminada exitosamente.\newline

Cualquier acción que no ocurra en ese orden será considerada
como fallo a anotar y, en caso de que haya expresiones
verbales negativas por parte del/la usuario/a, se anotará
dentro de la métrica correspondiente.\newline

La tarea deberá finalizarse en menos de un minuto.

\newpage

\section{Pregunta 4}

\textit{Entrega todos los documentos indicados y una breve
explicación de qué es cada uno de ellos (media página).}

\subsection{Guión}

\textbf{Explicación:} Se trata del guión que utilizaremos
durante el 
test con usuarios de la página de Freixenet. En este se
incluye la introducción a la prueba, explicando un poco el
motivo por el cual se está realizando este test, dando un
poco de trasfondo sobre lo que es Freixenet por si existe
algún/a participante que no conozca la marca y exponiendo la
duración de la prueba, junto con la media de tiempo que se
espera dedicar a cada tarea.\newline

Seguido de ello, se irán anunciando las tareas, una por una,
a medida que se vayan completando las actividades a
realizar.\newline

\textit{Buenos días y bienvenidos. Hoy realizaremos una prueba de
usabilidad de la página freixenet.es. Esta es la página
oficial de la marca de venta de cavas y vinos.
Principalmente, se dedican a esto, pero también hacen otras
actividades como organizar experiencias y eventos en sus
bodegas.}\newline

\textit{Esta prueba tiene una duración total de treinta minutos, en
los cuales haremos tres actividades, de aproximadamente tres
minutos, cada una.}\newline

\textit{Estas actividades tendrán que ver con diferentes
funcionalidades que tiene la página y, una vez las
completemos, o pase el tiempo requerido para hacerlas, o
queráis abandonar, pasaremos a la siguiente. Os dejaremos
listo un buscador web para cada tarea, y deberéis empezar la
actividad desde ahí.}\newline

\textit{Recordad que si tenéis alguna duda, la podéis consultar al
momento con el entrevistador. Os pedimos intentar verbalizar
lo máximo posible todo lo que se os vaya pasando por la
cabeza durante la realización de las pruebas, para tener un
mayor contexto del procedimiento utilizado para resolver la
tarea.}\newline

\textit{La primera actividad es la siguiente:
Entra en la página freixenet.es y busca la página de compra
del Cava Reserva Real.}\newline

\newpage

\textit{La siguiente actividad es:}\newline

\textit{Entra en la página freixenet.es
y haz una reserva para una cata de cavas y quesos. La
reserva será a nombre de Pablo Vallecas y pondrás el correo
pablovalle8756@gmail.com. Esta reserva será para tres
adultos, de horario de mañana. La primera preferencia de
idioma será el castellano y la segunda, el catalán y la
fecha de visita es para el 5 de diciembre de 2023. El resto
de información se puede añadir de la manera que tú creas más
adiente.}\newline

\textit{Y, por último, tenemos la siguiente actividad:
Busca, dentro de la página, información sobre cómo se formó
la empresa y cuál es su historia.}\newline

\textit{Para finalizar la actividad, nos gustaría agradecer la
participación de todas las personas que han acudido. Gracias
a vuestras contribuciones, seremos capaces de mejorar la
experiencia de usuario de la página. Queríamos pedir a los
participantes la realización de una encuesta sobre la
experiencia que habéis tenido con la página en la ejecución
de las actividades que hemos hecho hoy. Una vez finalizada,
habríamos terminado las pruebas en su totalidad.}\newline

\newpage

\subsection{Documento de consentimiento informado}

\textbf{Explicación:} Se trata de un documento que todos/as los/as
participantes, conforme autorizan a participar en las
diferentes pruebas que se realizarán y también se autoriza
la recogida de datos personales con el fin de evaluar la
usabilidad del producto a probar.

\hspace{2cm}

\begin{center}
\textbf{CONSENTIMIENTO INFORMADO}
\end{center}

Yo, $\underset{\text{(Nombre
completo)}}{\underline{\hspace{6cm}}}$, con
DNI$\underset{\text{(DNI)}}{\underline{\hspace{3cm}}}$,
declaro estar en
conocimiento y acepto mi
colaboración para la tarea de evaluación de la usabilidad
del sitio web freixenet.es.\newline

Dicha aplicación será realizada por el alumno Oussama
Akachach Jouhrati (oakachach@uoc.edu) del máster en
Ingeniería Informática de la Universitat oberta de
Catalunya, en el contexto de la asignatura Ingeniería de la
Usabilidad.\newline

La administración del test requiere una única sesión, en la
que se realizarán las actividades, junto con una encuesta de
valoración general, al finalizarlas.\newline

La información recabada será utilizada sólo con fines
académicos y será confidencial, teniendo acceso a ésta solo
los estudiantes participantes y el equipo docente de la
asignatura.\newline

Esta colaboración tiene carácter absolutamente voluntario,
no estando en la obligación de responder si no se desea, ni
permanecer más tiempo del acordado.\newline

La responsable de este proceso es Mireia Ribera turró
(mriberatu@uoc.edu), profesora de la asignatura mencionada.
Ante cualquier duda, por favor comunicarse con ella.\newline

\hspace{2cm}

Nombre y firma del entrevistado:\newline

\hspace{2cm}

Nombre y forma del entrevistador:\newline

\hspace{2cm}

Barcelona, noviembre de 2023.

\newpage

\subsection{Descripción de las tareas}

\textbf{Explicación:} Esta hoja contiene las descripciones que se
darán a los usuarios para la realización de las diferentes
tareas de la prueba. Existe también una versión para los
entrevistadores, para tener a mano los criterios de
evaluación de cada tarea.

\subsubsection{Versión de los participantes:}

\textbf{Actividad 1:} Entra en la página freixenet.es y busca la
página de compra del Cava Reserva Real.\newline

\textbf{Actividad 2:} Entra en la página freixenet.es
y haz una reserva para una cata de cavas y quesos. La
reserva será a nombre de Pablo Vallecas y pondrás el correo
pablovalle8756@gmail.com. Esta reserva será para tres
adultos, de horario de mañana. La primera preferencia de
idioma será el castellano y la segunda, el catalán y la
fecha de visita es para el 5 de diciembre de 2023. El resto
de información se puede añadir de la manera que tú creas más
adiente.\newline

\textbf{Actividad 3:} Busca, dentro de la página, información sobre
cómo se formó la empresa y cuál es su historia.

\subsubsection{Versión de los entrevistadores:}

\textbf{Actividad 1:} El/la usuario/a debe entrar en la
página desde el buscador del navegador web duckduckgo.com,
seguido de ello, deberá buscar en el menú la opción ``Cavas
emblemáticos''. Consiguientemente, el/la usuario/a deberá
buscar en el listado de cavas la opción ``Reserva Real''. Por
último, el/la usuario/a tendrá que pulsar en el botón
``Comprar'' para finalizar la tarea.\newline

Cualquier acción que no ocurra en este orden será
considerada como fallo a anotar y, en caso de que haya
expresiones verbales negativas por parte del/la usuario/a,
se anotará dentro de la métrica correspondiente. La tarea
debe realizarse en menos de 1 minuto y treinta
segundos.\newline

\textbf{Actividad 2:} El/la usuario/a debe entrar a la
página desde el buscador google.com.
Seguido de ello, deberá buscar en el menú la opción
``Enoturismo - Actividades enoturísiticas''.
Consiguientemente, el/la usuario/a deberá acercar el ratón
al cuadro con el título ``Cata de cavas y chocolate'',
aparecerá visible el botón 'Información y reservas', que
deberá pulsar para que se abra el formulario de reservas.
Aquí, deberá introducir todos los datos que se le ha
indicado y, para los que no ha indicado, debe responder de
una manera coherente al tipo de campo seleccionado. La tarea
finalizará cuando pulse el botón ``Enviar'' y aparezca el
mensaje de confirmación en la pantalla del
formulario.\newline

La duración de esta tarea es de 5 minutos.\newline

Cualquier acción que no ocurra en este orden será
considerada como fallo a anotar y, en caso de que haya
expresiones verbales negativas por parte del/la usuario/a,
se anotará dentro de la métrica correspondiente.\newline

Además, por cada elemento del formulario escrito de manera
incorrecta según el tipo de campo contará como dos errores. De
la misma manera, un error que detecta el formulario contará
sólo como un error.\newline

En esta tarea se debe prestar especial atención a cómo el
usuario introduce el campo ``Fecha de la visita''. Si pregunta
sobre el formato o no entiende cómo introducir ese campo, lo
contaremos como expresión negativa del/la usuario/a.\newline

\textbf{Actividad 3:} El usuario/a deberá entrar a la página
desde el navegador bing.com. Seguido de ello, deberá navegar
por el menú, hasta encontrar el enlace ``Descubre Freixenet -
Una historia de excelencia''. Una vez dentro, si el usuario
considera que ya ha llegado, o pulsa en un enlace que lleve
a más detalles sobre los acontecimientos de ese año,
consideraríamos la tarea como terminada
exitosamente.\newline

Cualquier acción que no ocurra en este orden será
considerada como fallo a anotar y, en caso de que haya
expresiones verbales negativas por parte del/la usuario/a,
se anotará dentro de la métrica correspondiente.\newline

La tarea deberá finalizarse en menos de 1 minuto.\newline

\newpage

\subsection{Encuesta SUS}

\textbf{Explicación:} Encuesta que se realizará después de la prueba
para evaluar la experiencia general del usuario/a. Se ha
sustituido la palabra ``sistema'' por ``freixenet.es''.\newline

Puntúa del 1 al 5, siendo 1 ``Completamente en desacuerdo'' y
5 ``Completamente de acuerdo''.\newline

\begin{itemize}

\item Creo que me gustaría utilizar la página más
frecuentemente.

\item He encontrado la página innecesariamente compleja.

\item He creído que la página era fácil de usar.

\item Creo que necesitaría la ayuda de una persona técnica para
utilizar esta página.

\item He encontrado que las diversas funciones de esta página
estaban bien integradas.

\item He creído que había mucha inconsistencia en esta página.

\item Imagino que la mayoría de usuarios/as aprenderían a usar
esta página muy rápido.

\item He encontrado el uso de la página muy rebuscado.

\item Me he sentido muy seguro usando la página.

\item He necesitado aprender muchas cosas antes de poder
avanzar con esta página. 

\end{itemize}

\newpage

\subsection{Hojas de recogida de datos}

\textbf{Explicación:} Este documento se utilizará para anotar la
información relevante para cada tarea. Habrá un documento
por cada usuario entrevistado y este contendrá una sección
para anotar las métricas utilizadas, junto con otra para
anotar observaciones adicionales, no capturadas en la fase
de planificación, que podrían ser interesantes a tener en cuenta
más adelante.\newline

\textit{NOTA: Todos los tiempos se anotarán en segundos.}

\subsubsection{Tarea 1}

Tiempo de finalización:

Número de errores cometidos:

Expresiones de descontento:

\subsubsection{Tarea 2}

Tiempo de finalización:

Número de errores cometidos:

Expresiones de descontento:

\subsubsection{Tarea 3}

Tiempo de finalización:

Número de errores cometidos:

Expresiones de descontento:

\subsubsection{Evaluación final}

Tiempo de finalización:

Número de errores cometidos:

Expresiones de descontento:

Al finalizar las tareas:

Tareas completadas totales: 

Errores cometidos totales:

Expresiones de descontento totales:\newline

Porcentaje de éxito por tarea:
\(\frac{TareasCompletadas}{3} = \)\newline

Tiempo medio de finalización de una tarea:
\(\frac{SumaTiemposFinalizacion}{3} = \)\newline

Ratio de errores por tarea:
\(\frac{SumaErroresCometidos}{3} = \)\newline

Expresiónes de descontento por tarea:
\(\frac{SumaExpresionesDescontento}{3} = \)\newline

\newpage

\begin{thebibliography}{X}

\item \textit{Design Toolkit | Test con usuarios.} (n.d.).
Fundación UOC. Noviembre 2023, de
http://design-toolkit.recursos.uoc.edu/es/test-con-usuarios/

\item Ribera, M. \textit{Caso de estudio resuelto.
Planificación de un test de usuarios}. Fundación UOC.
Noviembre 2023, de
https://materials.campus.uoc.edu/daisy/Materials/PID\textunderscore
00296954/html5/
PID\textunderscore 00296954.html\#w31aac13c11c11

\item Ribera Turró, M. (2023). \textit{Orientaciones
prácticas para el test de usabilidad. Métricas e
instrumentos.} [Recursi de
aprendizaje textual]. 1.ª ed. Barcelona: Fundació
Universitat Oberta de Catalunya (FUOC).

\item Rubin, J., \& Chisnell, D. (2008). \textit{Handbook of
Usability Testing: How to Plan, Design, and Conduct
Effective Tests.} John Wiley \& Sons.

\item \textit{Using moderated usability testing - Service
Manual - GOV.UK.} (2017, October 3).
https://www.gov.uk/service-manual/user-research/using-moderated-usability-testing\#steps-to-follow

\end{thebibliography}

\end{document}
